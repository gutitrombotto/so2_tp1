% simple.tex - A simple article to illustrate document structure.

% Andrew Roberts - June 2003

\documentclass[10pt, a4paper,notitlepage]{article}
\usepackage[spanish]{babel}%Para el español
\usepackage[utf8]{inputenc}


%\usepackage{times}
\usepackage{color}
\usepackage[dvipsnames]{xcolor}
\usepackage{listings}
\usepackage{textcomp}
\usepackage{pdfpages}

\usepackage{verbatim}

\usepackage{amsmath}
\usepackage{courier} %--lisitngs
\usepackage{mathptmx} %-->TimesNewRoman

\usepackage{caption}
\DeclareCaptionFont{white}{\color{white}}
\definecolor{dark-gray}{cmyk}{0,0,0,0.7}
\DeclareCaptionFormat{listing}{\colorbox{dark-gray}{\parbox{\textwidth}{#1#2#3}}}
\captionsetup[lstlisting]{format=listing,labelfont={white,sf},textfont={white,sf}}

\usepackage{graphicx} % Required for including pictures

\usepackage{float} % Allows putting an [H] in \begin{figure} to specify the exact location of the figure
\usepackage{wrapfig} % Allows in-line images such as the example fish picture
\graphicspath{{Imagenes/}} % Specifies the directory where pictures are stored

\usepackage{xcolor}
\definecolor{lbcolor}{rgb}{0.9,0.9,0.9}
\lstset{
backgroundcolor=\color{gray!20!white},
    tabsize=2,
    literate=%
	{á}{{\'{a}}}1
    {é}{{\'{e}}}1
    {í}{{\'{i}}}1
    {ó}{{\'{o}}}1
    {ú}{{\'{u}}}1
    {ñ}{{\~{n}}}1
    {<}{{$<$}}1
    {>}{{$>$}}1,
%   rulecolor=,
    %language=C,
    %language=bash,
    		linewidth=12.2cm,
		belowcaptionskip=0.1\baselineskip,
		xleftmargin=\parindent,        
        %basicstyle=\scriptsize,
        basicstyle=\footnotesize\bfseries\ttfamily\color{black!80!white},
        upquote=true,
        %aboveskip={1.5\baselineskip},
        columns=fixed,
        showstringspaces=false,
        extendedchars=false,
        breaklines=true,
        prebreak = \raisebox{0ex}[0ex][0ex]{\ensuremath{\hookleftarrow}},
	%frame=lines,
        numbers=left,
        showtabs=false,
        showspaces=false,
        showstringspaces=false,
        identifierstyle=\ttfamily,
        commentstyle=\color{orange!80!black},
        keywordstyle=\bfseries\color[rgb]{0,0,1},
        %commentstyle=\bfseries\color[rgb]{0.026,0.112,0.095},
        %stringstyle=\bfseries\color[rgb]{0.627,0.126,0.941},
        stringstyle=\bfseries\color{green!50!black},
        numberstyle=\ttfamily\color[rgb]{0.205, 0.142, 0.73},
%        \lstdefinestyle{C++}{language=C++,style=numbers}’.
}
\lstdefinelanguage{linux_bash}
{ language=bash,%-->me baso en bash
alsoletter={-},%-->me habilita (-) en medio de los keywords
morekeywords={sudo,nano,apt-get,cd,ls,ln,mkdir,ifconfig,cp,wget,ppa:,add-apt-repository,install,a2ensite,service,curl},
%sensitive=false,
}
\lstdefinelanguage{cisco}
{
alsoletter={-},%-->me habilita (-) en medio de los keywords
keywords={enable,configure,terminal,interface,fastEthernet,ipv6,address,no,shutdown,route,running-config,startup-config,copy,exit,unicast-routing,brief},
%sensitive=false,
}

\renewcommand{\lstlistingname}{Código}
\renewcommand{\familydefault}{\rmdefault}
\pretolerance=2000
\tolerance=3000
\newcommand{\HRule}{\rule{\linewidth}{0.5mm}} % Defines a new command for the horizontal lines, change thickness here
\usepackage{hyperref}
\hypersetup{%
colorlinks=true,
urlcolor=blue,
urlbordercolor=blue,
pdfborderstyle={/S/U/W 1}%
}

\begin{document}

% Article top matter
{\center \large \textsf{Facultad de Ciencias Exactas, Físicas y Naturales, Laboratorio de Redes y
Comunicaciones, Redes de computadoras\\}}
\title{%\HRule \\[0.4cm]		
		{ \bfseries{Trabajo Práctico 1: Ruteo estático con IPv4 e IPv6.}}\\[0.4cm]
		%\HRule \\[1.5cm]
		} %\LaTeX is a macro for printing the Latex logo
\author{
\textsc{Trombotto}, Agustin  {\small \texttt{Mat:39071116}}\\
\href{mailto:gutitrombotto@gmail.com}{gutitrombotto@gmail.com}\\
}
%\affil{Facultad de Ciencias Exáctas, Físicas \& Naturales, Universidad Nacional de Córdoba, Argentina.}

%\date{\today}  %\today is replaced with the current date
%\maketitle
{\let\newpage\relax\maketitle}



\textbf{Palabras Clave:} IPv6, GNS3, ruteo estático.\\
%\clearpage

\section{Introducción}
Es un elemento muy importante del paper, a través de ella el lector se nutre de la información 
suficiente para  comprender y evaluar por qué fue necesario realizar el estudio.
La introducción lleva al lector desde lo que ya sabe a lo que el investigador quiere decirle. Al
terminar de leer la introducción, el lector estará persuadido de que hay un problema
importante que abordar y comprenderá el contexto del mensaje principal que se le transmitió.
Por lo tanto, la introducción debe concentrar, con fluidez y precisión, de manera discursiva, los
principales elementos del problema y de la investigación, permitiendo al lector familiarizarse
con ellos.
He aquí una lista de aspectos que se deben tener en cuenta en la preparación de la introducción:
\subsubsection{Propósito}
Se desarrolla un sistema de comunicacion mediante sockets implementados en lenguaje c. Clientes se conectan a un servidor el cual debe transferir informacion sobre las estaciones meteorologicas
\subsubsection{Ambito del Sistema}
\subsubsection{Definiciones, Acrónimos y Abreviaturas} 
\subsubsection{Referencias}

\subsubsection{Descripción General del Documento}
Se detalla, en este documento los objetivos, el diseño e implementacion, los bugs del sistema y las conclusiones del mismo.
\section{Descripción General}
\subsubsection{Perspectiva del Producto }
El producto consiste en la ejecucion de un programa cliente y un programa servidor.
Una vez ejecutados, el cliente se conecta al servidor y le podŕa solicitar informacion sobre las estaciones meteorologicas en cuestion al servidor mediante las funciones proveídas por el mismo.
El cliente puede ser ejecutado en cualquier dispositivo con una interfaz de comunicaciones. Se le debe establecer previamente el IP del dispositivo a la cual se conectará
\subsubsection{Funciones del Producto}
\subsubsection{Características de los Usuarios }

\subsubsection{Suposiciones y Dependencias}

\subsubsection{Requisitos Futuros}
La implementacion del sistema posee funcionalidades pendientes a saber:
\begin{itemize}
	\item Envio de prompt por parte del servidor al cliente 
	\item Respuesta del servidor con la informacion correcta segun la funcion a ejecutar
	\item Transladar y testear en código en un servidor remoto 
	
\end{itemize}
\section{Requisitos Específicos}
\begin{itemize}
	\item Todos los procesos deben ser mono-thread
	\item Uso de Cppcheck y la compilación con el uso de las flags de warning -Werror, -Wall y -pedantic
	\item Proveer los archivos fuente, así como cualquier otro archivo asociado
	a la compilación
	
\end{itemize}

\subsubsection{Interfaces Externas}
\subsubsection{Funciones}
El servidor debe contar con las siguientes funcionalidades:
\begin{itemize}
	\item \textbf{listar:} muestra un listado de todas las estaciones que hay en la “base de datos”,
	y muestra de que censores tiene datos.
	\item \textbf{descargar no\_estación:} descarga un archivo con todos los datos de
	no\_estación.
	\item \textbf{diario\_precipitacion no\_estación:} muestra el acumulado diario de la variable
	precipitación de no\_estación (no\_día: acumnulado mm).
	\item\textbf{ mensual\_precipitacion no\_estación:} muestra el acumulado mensual de la
	variable precipitación (no\_día: acumnulado mm).
	\item \textbf{promedio variable:} muestra el promedio de todas las muestras de la variable
	de cada estación (no\_estacion: promedio.
	\item \textbf{desconectar:} termina la sesión del usuario.
\end{itemize}

Ademas el servidor debe proveer al cliente de un prompt de la forma:
\textbf{usuario@servidor}

Tambien un sistema de autenticacion, donde el usuario deberá incresar su contraseña y el servidor validar si pertenece a un usuario apto para ingresar.
\subsubsection{Requisitos de Rendimiento}
El sistema deberá tener la opcion de ser ejecutado en un servidor remoto en el cual se establece una direccion ip, se levanta un socket en el puerto 6020, y el cliente se deberá poder conectar a dicha ip con dicho puerto desde la misma red.
\subsubsection{Restricciones de Diseño}
\begin{itemize}
	\item El sistema no podrá contar con un sistema de gestion de base de datos.
	\item El sistema se correrá en un equipo que cuenta con las herramientas típicas de consola para el desarrollo de programas (Ej: gcc,make)
\end{itemize}


\subsubsection{Atributos del Sistema}
\subsubsection{Otros Requisitos }
Para la presentacion final, el desarrollo se correrá en una INTEL Galileo V1, sobre la cual se comprobará el prototipo.
\section{ Diseño de solución}

%\section*{Sección sin numeración}
%Este es el cuerpo de la sección del trabajo.
%\subsection*{Subsección sin numeración}
%Este es el cuerpo de la sub-sección de la sección anterior.
%\subsubsection*{Subsubsección sin numeración}
%Este es el cuerpo de la sub-sección de la subsección anterior.



\section{Implementación y Resultados}
En esta sección se colocan dos imágenes

\section{Reporte de Bugs}

\section{Conclusiones}

\section{Apendice}

\begin{thebibliography}{100} % 100 is a random guess of the total number of
%references
\bibitem{Comer2014} Douglas E. Comer, ``Internetworking With TCP/IP," \emph{Vol I: Principles, Protocols, and Arquitecture}, pp. 473-480, Sixth Edition 2014.
\end{thebibliography}


\end{document}